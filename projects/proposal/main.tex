\documentclass{article} % For LaTeX2e
\usepackage{nips15submit_e,times}
\usepackage{hyperref}
\usepackage{url}
\usepackage{graphicx}
%\documentstyle[nips14submit_09,times,art10]{article} % For LaTeX 2.09


\title{Predicting Publication Conference with Deep Learning}


\author{
Jingyuan Liu\\
AndrewId: jingyual\\
\texttt{jingyual@andrew.cmu.edu} \\
}


\newcommand{\fix}{\marginpar{FIX}}
\newcommand{\new}{\marginpar{NEW}}


\nipsfinalcopy % Uncomment for camera-ready version


\begin{document}
\maketitle


\section{Introduction}
When researchers are writing a paper, they will decide which conference it
should be submitted to. Submitting a paper to a proper conference is a very
important. A conference with close research topics and good reputation is likely to
make the paper more influential. Therefore, in general, good conferences means
good quality of a paper in related research topics.

Predicting the conference of a paper is an interesting and challenging problem.
Information of the paper, like words, authors, organizations and time are highly related to
the paper conference information. Conference contains several latent research
topics, which will overlap with the latents topics of paper submitted to it. If
I could properly represent the latent research topic information and capture the
relation of the latent information and conference label, then I can solve the
predicting challenge. To predict the conference for a publication, the
conference will be treated as the classification label, and other information
of the publication will be treated as features, for example, the authors,
the organizations, and paper time.

Besides, prediction method could extend to a more interesting recommendation model.
If I use generative classification models for the problem, I can get all parameters
after training. Then I can use the trained model to recommend conferences for a
new paper to submit. Given the information of all attributes of a paper, which
could be transferred to features of the model, like authors, organizations, and
time, I can use the trained model to get an output of the conference. If the
model can get impressive performance, this model could be quite useful to
provide suggestions for those new researchers.

\section{Data Description}
The dataset is a mixture collection of different papers in
different conferences by different authors published at different time. I
treated the conferences of those published paper as the label for the paper
instances. Other attributes of a paper as features. Specifically:

\textbf{Authors}: authors of a paper is a very important attribute of a paper. I
can use an indicator vector to encode this attribute to a vector as input
feature. At first, I assign every different author a index. Then for each paper, I would
put 1 to the position of the index for every author in the paper, otherwise 0.

\textbf{Organizations}: it is similar to authors.

\textbf{Time}: time is also very import attribute for a paper. The research topics of a
conference would change over time. Therefore, when predicting conference for a
paper, I need to take the time into consideration. I could use the year as
the input of the feature.

\textbf{Words}: words in a paper is likely the most important information for mining the
latent topics of a paper. Therefore, words features are very important for
prediction of a paper conference. Most NLP problems use "bag of words" method to
encode words as features.

I will use a dataset obtained online for this model. The dataset was published
by Prof. Jie Tang from Tsinghua University, which contains over 20+ million
papers. After preprocess, the dataset currently contains:

\textbf{paper}: 1.5m +

\textbf{author}: 60k +

\textbf{conference}: 60k +

\textbf{year range}: 1951 $\sim$ 2014

\textbf{organization}: 20k +

\textbf{words}: 600k +

The dataset is currently too large for a normal classification problem. I will
further preprocess the dataset to get a more reasonable and feasible dataset
with smaller size.

\section{Baseline Methods}
The proposal project is a classification problem in the supervise learning
filed. There are many traddtional methods could be used to solve this challenge.
For example, Naive Bayes, and Decision Trees, could be used to solve this problem.

Naive Bayes is a generative classification model. I can use words of a paper as
feature input and conference token as label for training. Decision Tree is a
discriminative classification model. I can use decision tree to generate rules
and ``split'' data into different groups for the classification task.


\section{Proposed Model}
I want to use deep learning to solve this problem. Deep learning is currently
becoming more and more popular and powerful in several machine learning and
pattern recognition related research fileds, for example, recognizing cat faces
for computer vision, and word2vec for natural language processing. With large
datasets, deep learning methods usually perform obviously better than
traditional classification models for supervised learning.

First, I can use word2vec to encode word features for words in a paper.
Traditional classification method use the bag of words model to represent word
as features, which may lose the information of word orders and word relations.
Word2vec is a model proposed by Mikolov, which is a distributed representation
of word and phrases. This model could encode word to a vector, which performed
better than bag of words in some cases.

Next, I would use a neural network model for classification. I would at first
build simple layers for the function. For example, for the first layer, I would
use the sum of feature input and an activation function. For the second layer, I
would use a logistic basis. And at last, I would use a softmax function to make
the output a distribution. For model training, I would use Backprop and SGD to
optimize model parameters iteratively.

I would use early stopping and some other method to avoid overfitting during
training for NN model. After training, I would conduct cross validation with the
proposed model and compared to baseline models. I will give a detailed analysis
of those models mentioned above.
\end{document}
